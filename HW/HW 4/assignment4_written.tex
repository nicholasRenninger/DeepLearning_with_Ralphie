\documentclass{article}
\usepackage{amsfonts}
\usepackage{amsmath}
\usepackage{mathrsfs}
\usepackage{tikz}
\usepackage{epigraph}
\usepackage{enumitem}
\usepackage{multicol}
\usepackage{geometry}
\usepackage{marginnote}
\usepackage{makecell}
\begin{document}
\pagenumbering{gobble}
\newpage
\begin{center}
  \bfseries\large
  Assignment 4, Written Part
\end{center}

\noindent

\vspace{20mm}

Please turn in the answers to this written part of assignment 4 by either 
 \begin{itemize}
     \item Typesetting your answers inline with LaTeX (.tex file provided).
     \item Write out your answers with tablet / stylus, and submit the annotated pdf.
     \item Print out the assignment, write answers by hand, and scan / photograph your work.  The image must be clearly legible, and all pages must be combined into one file.
     \item For the written response questions, clearly justify all conclusions to receive full credit. A correct answer with no supporting work will receive no credit.
 \end{itemize}

\newpage

1. Consider a linear model for classification in which we use a logistic activation, but instead of cross-entropy loss, we use squared error loss. Assume a 1-dimensional input $x$, a single weight $w$ and an outcome $y_i\in\{0,1\}$. We will ignore the intercept term.
\begin{align*}
a_i &= wx_i \\
p_{i}&=\mbox{logistic}\left(a_i\right) \\
l_{i}&=\left(y_i-p_{i}\right)^{2}
\end{align*}
Recall that $\mbox{logistic}\left(u\right)=\frac{1}{1+e^{-u}}$.
Calculate the following:

{
\renewcommand{\theenumi}{\alph{enumi}}
\begin{enumerate}
\item  $\dfrac{dl_i}{dp_i}$ \\
\item $\dfrac{dp_i}{da_i}$, as a function of $a_i$ \\
\item $\dfrac{dp_i}{da_i}$, rewritten as a function of $p_i$ only \\
\item $\dfrac{da_i}{dw}$ \\
\item $\dfrac{dl_i}{dw}$ \\
\item Assume that $y_i = 1$. What is $\lim_{p\rightarrow0}$ $\frac{dl_i}{dw}$? Is this good or bad for learning? Explain why.
\end{enumerate}
}

\newpage
2. Consider a linear model for classification based on the hinge loss, with a penalty for weight magnitude. This is the basic support vector machine (don't worry if you haven't studied it). Unlike question 1, we will now assume that $y_i\in\{-1, 1\}$. Again, assume a single input variable $x_i$, and ignore the intercept term.
\begin{align*}
a_i &= wx_i \\
l_i &= \max (0, 1-y_i a_i) + w^2 \\
\end{align*}

Calculate the following: \\

{
\renewcommand{\theenumi}{\alph{enumi}}
\begin{enumerate}
\item \begin{tabular}{l l}$\dfrac{\partial l_i}{\partial a_i}$ & \makecell[l]{[Note: This technically should be a subgradient.  Only worry about the two cases of \\ $y_i a_i < 1$ and $y_i a_i > 1$ . Don't worry about the non-differentiable point where $y_ia_i = 1$.]}\end{tabular}
\item $\dfrac{dl_i}{dw}$ [Again, there are two cases.]
\item Assume that $y_i = 1$.  What is update rule for $w$ for stochastic gradient descent?
\item Contrast this rule with the update rule for the perceptron.
\end{enumerate}
}
\newpage
3. \begin{center}
$y = x + \frac{1}{wx + b}$
\end{center}
Draw the computation graph for calculating $y$ from $x, w$ and $b$, Fill in the blanks for the reverse mode AD table at $x=0.3,w = 0.5,b=0.1$
\begin{center}
Part 1 - Computation Graph
\end{center}
\vspace{150pt}
\begin{center}
Forward Primal Trace
\end{center}
\begin{center}
\begin{tabular}{ c l l }
$v_{-2}$ & $= x$ & $= 0.3$  \\
$v_{-1}$ & $= w$ & $= 0.5$  \\
$v_{0}$ & $= b$ & $= 0.1$ \\
\hline
$v_1$ & $= v_{-2} v_{-1}$ & $= 0.15$ \\ 
$v_2$ & $= v_1 + v_{0}$ & $= 0.25$ \\
$v_3$ & $= \frac{1}{v_2}$ & $= 4$ \\ 
$v_4$ & $= v_{-2} + v_3$ & $= 4.3$ \\ 
\hline
$y$ & $= v_4$ & $= 4.3$ \\
\end{tabular}
\end{center}
\begin{center}

Part 2 - Reverse Adjoint Trace
\end{center}
\begin{center}
\begin{tabular}{ c l l }
$\Bar{v_{-2}}$ & = \hspace{200pt} & $=$  \\
$\Bar{v_{-1}}$ & = \hspace{200pt} & $=$  \\
$\Bar{v_0}$ & = \hspace{200pt} & $= $  \\
\hline
$\Bar{v_1}$ & = \hspace{200pt} & $=$ \\
$\Bar{v_2}$ & = \hspace{200pt}  & $=$ \\
$\Bar{v_3}$ & = \hspace{200pt} & $=$ \\
\hline
$\Bar{v_4}$ & = \hspace{200pt} & $=1$
\end{tabular}
\end{center}

%\newpage
%4. Suppose you are training a binary classifier using a 1-hidden layer neural network, with 50 neurons in the hidden layer, to classify images of size 28 $\times$ 28 with 3 channels. How many parameters does your model have (including bias parameters)? Explain your calculation.
\end{document}